\documentclass{article}
\usepackage[margin=0.5in]{geometry}
\usepackage[T1]{fontenc}
\usepackage{hyperref}
\usepackage{enumitem}
\usepackage{titlesec}
\titleformat{\section}{\Large\bfseries\filcenter}{}{0pt}{}
\titlespacing*{\subsection}{0pt}{0pt}{0pt}
\titlespacing*{\subsubsection}{10pt}{1pt}{1pt}

\setlength{\parskip}{-0.2em}
\renewcommand{\baselinestretch}{1.05}
\renewcommand{\labelitemi}{\scriptsize{$\bullet$}} % makes lists have a small-ish bullet
\renewcommand{\labelitemii}{\scriptsize{$\circ$}} % makes sub lists have a small-ish empty circle as a bullet
\pagenumbering{gobble} % removes page numbering
\begin{document}
	\begin{center}
		\section*{Alexander Fischer}
			\href{mailto:afischer@umass.edu}{afischer@umass.edu} | \href{https://github.com/AlexDFischer}{https://github.com/AlexDFischer} | \href{https://linkedin.com/in/AlexDFischer}{https://linkedin.com/in/AlexDFischer}
    \end{center}
	\subsection*{Education}
		\subsubsection*{University of Massachusetts, Amherst \hfill \normalfont \normalsize Fall 2016--Spring 2020}
    		\setlist{nolistsep}
			\begin{itemize}[noitemsep,leftmargin=40pt]
				\item \textbf{Majors}: Computer Science, Pure Mathematics. \textbf{GPA}: 4.0.
				%\item \textbf{Relevant Coursework} (computer science): Computer Systems Principles, Advanced Algorithms (graduate level), Artificial Intelligence, Machine Learning, More Advanced Algorithms (graduate level), Formal Language Theory, Reinforcement Learning (gradate level, ongoing), Neural Networks: a Modern Introduction (graduate level, ongoing)
				%\item \textbf{Relevant Coursework} (mathematics): Multivariable Calculus, Statistics, Linear Algebra, Differential Equations, Modern Analysis, Complex Variables, Discrete Structures, Mathematical Cryptography, Abstract Algebra I \& II.
				\item \textbf{Relevant Coursework} (computer science): Neural Networks (graduate level), Reinforcement Learning (graduate level), Advanced Algorithms (graduate level), More Advanced Algorithms (graduate level), Formal Language Theory, Artificial Intelligence, Machine Learning, Computer Systems Principles
				\item \textbf{Relevant Coursework} (mathematics): Multivariable Calculus, Statistics, Linear Algebra, Differential Equations, Modern Analysis, Complex Variables, Discrete Structures, Mathematical Cryptography, Abstract Algebra I \& II.
			\end{itemize}
			\subsection*{Academic Experience}
			\subsubsection*{Autonomous Mobile Robotics Laboratory, University of Massachusetts Amherst \hfill \normalfont \normalsize January 2018--present}
			\begin{itemize}[noitemsep,leftmargin=40pt]
				\item Performed original research on a novel algorithm for time-optimal control of omnidirectional robots and implemented that algorithm on real robots in C++.
				\item Published a second-author paper in the International Conference On Intelligent Robots and Systems, 2018.
				\item Wrote software to automatically calibrate latency values for robot motion.
			\end{itemize}
			\subsubsection*{Research Experience for Undergraduates, University of Miami \hfill \normalfont \normalsize Summer 2017}
			\begin{itemize}[noitemsep,leftmargin=40pt]
				\item Wrote software to analyze three dimensional images of mice optic nerves that were multiple gigabytes each, in order to assist medical researchers studying neuron regeneration.
				\item Implemented novel and existing computer vision algorithms in MATLAB and C++.
				\item My research poster is available at \href{http://www.cs.miami.edu/reu-cfs/2017/posters/FischerPublicPoster.pdf}{http://www.cs.miami.edu/reu-cfs/2017/posters/FischerPublicPoster.pdf}.
			\end{itemize}
			\subsection*{Publications}
			\begin{itemize}[noitemsep,leftmargin=40pt]
				\item David Balaban, Alexander Fischer, Joydeep Biswas (2018). A Real-Time Solver For Time-Optimal Control Of Omnidirectional Robots with Bounded Acceleration. In \textit{Intelligent Robots and Systems (IROS), IEEE/RSJ International Conference on}. Available: \href{https://arxiv.org/abs/1707.04617}{https://arxiv.org/abs/1707.04617}
			\end{itemize}
	\subsection*{Work Experience}
		\subsubsection*{Software Engineer Intern, Microsoft \hfill \normalfont \normalsize Summer 2018}
        	\begin{itemize}[noitemsep,leftmargin=40pt]
            	\item Added features to the Windows photo viewer and to the Photos Companion mobile app used to import photos from phones into a PC.
                \item Used C\# with UWP for the desktop application and C\# with Xamarin for the cross-platform mobile application.
                \item Designed and implemented new network protocol features to improve the photo transfer experience.
            \end{itemize}
        \subsubsection*{Teaching Assistant, University of Massachusetts Amherst \hfill \normalfont \normalsize January--December 2017}
        	\begin{itemize}[noitemsep,leftmargin=40pt]
            	\item Undergraduate TA for math class `Fundamental Concepts of Mathematics'. \hfill \normalfont \normalsize Fall 2017
                \begin{itemize}[noitemsep]
                	\item Taught discussion sections, held office hours, and graded homework assignments.
                \end{itemize}
                \item Undergraduate TA for computer science class `Programming with Data Structures'. \hfill \normalfont \normalsize Spring 2017
                \begin{itemize}[noitemsep]
                	\item Graded assignments from discussion sections.
                \end{itemize}
            \end{itemize}
	\subsection*{Skills}
		\begin{itemize}[noitemsep,leftmargin=40pt]
			\item \textbf{Programming languages}: C++, C, C\#, Java (including Android), Python.
			\item \textbf{Technologies}: PyTorch, Matplotlib, Git, Linux, Xamarin, \LaTeX.
		\end{itemize}
	\subsection*{Personal/Class Projects}
    	\begin{itemize}[noitemsep,leftmargin=40pt]
    		\item \textbf{LSTM transfer learning}: Came up with a way to perform transfer learning with stacked LSTM neural networks and implemented my ideas on text data for a final project in a graduate-level neural networks class. Used Python and PyTorch.
        	\item \textbf{Quadratic sieve}: Implemented quadratic sieve factoring algorithm as part of a group project in a mathematical cryptography class. Successfully factored 120 bit numbers in less than a day. Used C.
            \item \textbf{Chamberwell}: Android game published on the Google Play store where one tilts the screen to transport moving balls into the correct chambers. Used Java, Android Studio.
			\item \textbf{Mandelbrot set renderer}: Renders the Mandelbrot set with smooth coloring and multithreading. Used Java.
            \item \textbf{SPIRE autoenroller}: Continuously checks if a class is open on SPIRE, the course enrollment system at UMass, then automatically enrolls one in it if so. Used Java, Selenium.
        \end{itemize}
    \subsection*{Activities and Awards}
    	\begin{itemize}[noitemsep,leftmargin=40pt]
        	\item \textbf{Putnam Exam, 2017} (a national mathematics competition for undergraduate students): Scored 19 points, ranking in the top 17\% of the country.
        	\item \textbf{Jacob-Cohen-Killam Math Competition, 2017} (competition for University of Massachusetts students): won second place, including a \$1000 prize.
            \item \textbf{Hack Harvard, 2017}: Won best IoT hack for a voice controlled robotic drink mixer built with Amazon Alexa.
            \item \textbf{Hack Holyoke, 2016}: Won best hardware hack for a bike lock that could be controlled from a phone via bluetooth.
        \end{itemize}
%    \subsection*{Contact Information}
%	    \begin{itemize}
%	    	\item \textbf{Email}: \href{mailto:afischer@umass.edu}{afischer@umass.edu}
%	    	\item \textbf{Phone}: 508-446-0400
%	    	\item \textbf{Home address}: 49 Beaver Street, Franklin, MA, 02038. \textbf{School Address}: Johnson Hall, Room 312, 380 Thatcher Road, Amherst, MA 01003-9359.
%	    \end{itemize}
\end{document}