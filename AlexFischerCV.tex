\documentclass{article}
\usepackage[margin=0.75in]{geometry}
\usepackage[T1]{fontenc}
\usepackage{xurl}
%\PassOptionsToPackage{hyphens}{url}\usepackage{hyperref}
\usepackage{hyperref}
\usepackage{url}
\usepackage{doi}

\usepackage{enumitem}
\usepackage{titlesec}
\titleformat{\section}{\Large\bfseries\filcenter}{}{0pt}{}
%\titlespacing*{\subsection}{0pt}{0pt}{0pt}
%\titlespacing*{\subsubsection}{12pt}{4pt}{0pt}

\setlength{\parskip}{0em}
\renewcommand{\baselinestretch}{1.0}
\renewcommand{\labelitemi}{\scriptsize{$\bullet$}} % makes lists have a small-ish bullet
\renewcommand{\labelitemii}{\scriptsize{$\circ$}} % makes sub lists have a small-ish empty circle as a bullet
\pagenumbering{gobble} % removes page numbering

\renewcommand{\refname}{\vspace{-24pt}} % removes "References" title at start of bibliography

\begin{document}
	\begin{center}
		\section*{Alex Fischer}
			\href{https://alexfischer.science}{https://alexfischer.science} |
			\href{mailto:alexander.fischer3@gmail.com}{alexander.fischer3@gmail.com} | \href{https://scholar.google.com/citations?user=CfhWaI4AAAAJ&hl=en}{Google Scholar profile: bit.ly/3Q1p6eV}
    \end{center}
	\subsection*{Education}
		\subsubsection*{University of New Mexico \hfill \normalfont \normalsize Fall 2022--present}
			\begin{itemize}
				\item Physics PhD student.
				\item In Center for Quantum Information and Control (CQuIC)---research interest is quantum information and quantum computing.
			\end{itemize}
		\subsubsection*{University of Massachusetts, Amherst \hfill \normalfont \normalsize Fall 2016--Spring 2020}
			\begin{itemize}
				\item Graduated with 2 Bachelor of Science degrees in Computer Science and Pure Mathematics.
				\item GPA: 3.98.
			\end{itemize}
	\subsection*{Publications}
		\nocite{*}
		\bibliographystyle{quantum}
		\bibliography{./bib}{}
	\subsection*{Talks given}
		\begin{itemize}
			\item 2021 IEEE Conference on Quantum Computing and Engineering (QCE21), technical paper talk, ``Distributing Graph States Across Quantum Networks''.
			\item 2020 Workshop on Quantum Network Science (NetSci 2020 satellite workshop), flash talk, ``Distributing Graph States Across Quantum Networks''.
		\end{itemize}
	\subsection*{Research experience}
		\subsubsection*{Akimasa Miyake Research Group, University of New Mexico \hfill \normalfont \normalsize January 2023--present}
			\begin{itemize}
				\item Researching quantum error correction.
			\end{itemize}
		\subsubsection*{Quantum Networking Group, University of Massachusetts Amherst \hfill \normalfont \normalsize January 2020--October 2021}
			\begin{itemize}
				\item Devised new algorithm for preparing graph states in a quantum network.
				\item Proved our algorithm has better performance than that of prior work on the same problem.
				\item Work appeared as paper in 2021 IEEE International Conference on Quantum Computing and Engineering (QCE2021), in poster session of QCE2020, and in flash talk in Workshop on Quantum Network Science (NetSci 2020 Satellite Workshop).
			\end{itemize}
		\subsubsection*{Autonomous Mobile Robotics Laboratory, University of Massachusetts Amherst \hfill \normalfont \normalsize January 2018--May 2019}
			\begin{itemize}
				\item Modified novel control algorithm for time-optimal control of omnidirectional robots to improve algorithm's stability with respect to noisy robot motion.
				\item Implemented that algorithm in real time on real robots in C++.
				\item Work appeared as paper (second-author) in International Conference On Intelligent Robots and Systems, 2018.
				\item Wrote software to automatically calibrate latency values for robot motion.
			\end{itemize}
		\subsubsection*{Research Experience for Undergraduates, University of Miami \hfill \normalfont \normalsize Summer 2017}
			\begin{itemize}
				\item Wrote software to analyze three dimensional images of mice optic nerves that were multiple gigabytes each, in order to assist medical researchers studying neuron regeneration.
				\item Implemented novel and existing computer vision algorithms in MATLAB and C++.
				\item Poster available at \href{http://cs.miami.edu/reu-cfs/2017/posters/FischerPublicPoster.pdf}{cs.miami.edu/reu-cfs/2017/posters/FischerPublicPoster.pdf}.
			\end{itemize}
	
	\subsection*{Industry experience}
		\subsubsection*{Software Engineer, Microsoft \hfill \normalfont \normalsize August 2020--present}
		\begin{itemize}
			\item Microsoft AI Development \& Acceleration Program (MAIDAP)---a rotation program for new graduates.
			\item Rotating between different teams every 6 months within Azure cloud computing service.
			\item Used several different programming languages (C\#, Python, Typescript).
		\end{itemize}
		\subsubsection*{Software Engineer Intern, Microsoft \hfill \normalfont \normalsize Summer 2019}
		\begin{itemize}
			\item Improved an internal tool used to analyze customer satisfaction data gathered from Office 365 customer surveys.
			\item Full stack development with C\# on ASP.NET, SQL, Typescript, and React.
		\end{itemize}
		\subsubsection*{Software Engineer Intern, Microsoft \hfill \normalfont \normalsize Summer 2018}
		\begin{itemize}
			\item Added features to the Windows photo viewer and to the Photos Companion mobile app.
			\item Used C\# with UWP for the desktop application and C\# with Xamarin for the cross-platform mobile application.
			\item Designed and implemented new network protocol features to improve the photo transfer experience.
		\end{itemize}
	
	\subsection*{Teaching experience}
		\subsubsection*{Teaching Assistant, University of New Mexico \hfill \normalfont \normalsize August 2022--present}
		\begin{itemize}
			\item Lab TA for introductory physics courses `Survey of Physics' and 'Physics of Music':
			\begin{itemize}
				\item Set up and ran lab activities.
				\item Graded lab reports.
			\end{itemize}
			\item TA for Computer Science class `Programming with Data Structures': graded homework.
		\end{itemize}
		\subsubsection*{Teaching Assistant, University of Massachusetts Amherst \hfill \normalfont \normalsize January--December 2017}
			\begin{itemize}
			\item TA for 300 level Mathematics class `Fundamental Concepts of Mathematics' (intro to proof-based mathematics):
			\begin{itemize}
				\item Planned \& ran my own discussion sections.
				\item Held office hours.
				\item Graded exams and homework.
			\end{itemize}
			\item TA for Computer Science class `Programming with Data Structures': graded homework.
			\end{itemize}
	\subsection*{Grants awarded}
		\subsubsection*{Sara Corrie Grant, National Speleological Society, 2023}
			\begin{itemize}
				\item \$1000 awarded by national-level caving organization.
				\item Grant supported gear, transportation, and food costs for a 13 person cave exploration expedition on Prince of Wales Island, Alaska, that I helped organize.
			\end{itemize}
		\subsubsection*{Northern Rocky Mountain Grotto Small Grant, 2023}
			\begin{itemize}
				\item \$500 awarded by regional-level caving organization.
				\item Grant supported gear, transportation, and food costs for a 13 person cave exploration expedition on Prince of Wales Island, Alaska, that I helped organize.
			\end{itemize}
%	\subsection*{Awards}
%	\begin{itemize}[noitemsep,leftmargin=40pt]
%		\item \textbf{Putnam Exam, 2017} (a national mathematics competition for undergraduate students): Scored 19 points, ranking in the top 17\% of the country.
%		\item \textbf{Jacob-Cohen-Killam Math Competition, 2017} (competition for University of Massachusetts students): won second place, including a \$1000 prize.
%	\end{itemize}

%	\subsection*{Technical Skills}
%		\begin{itemize}[noitemsep,leftmargin=40pt]
%			\item \textbf{Programming languages}: C++, C, C\#, Java, Python.
%			\item \textbf{Technologies}: PyTorch, Matplotlib, Git, Azure, Linux, \LaTeX.
%		\end{itemize}

%	\subsection*{Personal/Class Projects}
%    	\begin{itemize}[noitemsep,leftmargin=40pt]
%    		\item \textbf{LSTM transfer learning}: Came up with a way to perform transfer learning with stacked LSTM neural networks and implemented my ideas on text data for a final project in a graduate-level neural networks class. Used Python and PyTorch. Writeup at \href{https://tinyurl.com/lstm-transfer}{https://tinyurl.com/lstm-transfer}
%        	\item \textbf{Quadratic sieve}: Implemented quadratic sieve factoring algorithm as part of a group project in a mathematical cryptography class. Successfully factored 120 bit numbers in less than a day. Used C.
%            \item \textbf{Chamberwell}: Android game published on the Google Play store where one tilts the screen to transport moving balls into the correct chambers. Used Java, Android Studio.
%			\item \textbf{Mandelbrot set renderer}: Renders the Mandelbrot set with smooth coloring and multithreading. Used Java.
%            \item \textbf{SPIRE autoenroller}: Continuously checks if a class is open on SPIRE, the course enrollment system at UMass, then automatically enrolls one in it if so. Used Java, Selenium.
%        \end{itemize}

%    \subsection*{Contact Information}
%	    \begin{itemize}
%	    	\item \textbf{Email}: \href{mailto:afischer@umass.edu}{afischer@umass.edu}
%	    	\item \textbf{Phone}: 508-446-0400
%	    	\item \textbf{Home address}: 49 Beaver Street, Franklin, MA, 02038. \textbf{School Address}: Johnson Hall, Room 312, 380 Thatcher Road, Amherst, MA 01003-9359.
%	    \end{itemize}
\end{document}