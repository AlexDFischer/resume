\documentclass{article}
\usepackage{hyperref}
\usepackage[margin=1in]{geometry}
\title{}
\author{Alex Fischer}
\date{}

\pagenumbering{gobble} % remove page number

\begin{document}
	\maketitle
	\vspace{-3em}
	\begin{center}
		\href{mailto:alexfischer97@unm.edu}{alexfischer97@unm.edu}
	\end{center}
	
	To those at Los Alamos National Laboratory,

	My research interests can be summarized as, broadly speaking, applied and computational physics and math. I am interested in both theoretical and experimental physics, and I am particularly attracted to problems that involve both a theoretical/mathematical component and an experimental component. Lately, this applied physics interest has manifested itself as a more specific interest in quantum information science; this field has that combination of heavy theory and math that goes along with the practical experimental results. However, I am still interested in applied physics broadly and am interested in working anywhere my skills would be useful.
	
	I have experience in research and in industry that would qualify me to work at Los Alamos National Laboratory. My first research experience was a Research Experience for Undergraduates (REU) program at the University of Miami the summer after my freshman year. I assisted researchers at the university's medical school studying neuron regeneration by writing software to analyze massive 3D images of mice optic nerves. There I gained experience writing high performance code that processed large amounts of data and ran remotely on dedicated clusters, which would be useful for any simulation work I would do at Los Alamos. 
	
	I also worked at a robotics laboratory (the \href{https://amrl.cs.umass.edu/}{Autonomous Mobile Robotics Laboratory}) at the University of Massachusetts, where I did my undergrad. My project there was developing a new optimal control algorithm specific to the dynamics of our omni-wheeled robots in order to time-optimally move them between position-velocity states. We published these results as \href{https://arxiv.org/abs/1707.04617}{a paper} (I was second author) in the 2018 IEEE Conference on Intelligent Robots and Systems. I enjoyed that this work combined interesting mathematics/theory (specifically, the functional analysis tools that were used to get optimal control results), simulation work to develop the implementation of our optimal control algorithm, and the experimental work to quantify how that implementation worked on our robots. I look forward to combining these aspects of research at Los Alamos: theory/math, simulation, and experimental work.

	I also had the opportunity to work in a quantum networking theory group at UMass. There my research was focused on a specific state preparation task in an idealized noise-free model of a quantum network. We developed a new quantum algorithm for this task and proved that it outperformed the best known algorithm for this problem from prior work based on several metrics. We published \href{https://arxiv.org/abs/2009.10888}{a paper with our results} in the 2021 IEEE Conference on Quantum Computing and Engineering. This is when I discovered my interest in quantum information and decided that it is the main research field I want to pursue.
	
	After I graduated from my undergrad, I worked for 2 years at Microsoft as a software engineer. During my time at Microsoft (and during my computer science education in my undergrad), I gained programming skills that will prove useful for experimental physics work and data analysis. While working on Azure (Microsoft's cloud computing service) backend, I gained more experience writing high performance code that processes large amounts of data.
	
	All of these experiences would make me a good fit for quantum information research at Los Alamos. I would be looking to start research at Los Alamos some time in the next year, ideally---anywhere from spring 2023 to fall 2023.
\end{document}